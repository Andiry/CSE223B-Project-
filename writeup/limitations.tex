
The system suffers from a number of limitations that could be improved on in the
future. The greatest issue in AJFS at present is that operations are queued for
execution on the receiving server, rather than on the sending one. This was
chosen for simplicity, as AJFS is able to leverage the buffered socket
connection to queue requests. This approach requires, however, that every
request be immediately forwarded on to other hosts.

Operations that need to be fast (for instance, \texttt{write()}) thus suffer, as
they need to be fully transferred to other hosts before the operation is able to
return to the user. Even though the actual operation execution is asynchronous,
the data transmission often dominates the time it takes to perform the
operation, and that is fully synchronous.

This results in bandwidth that is inversely proportional to the number of nodes
that are active in the system. If AJFS queued outgoing requests and handled
transmitting of the data asynchronously, bandwidth would be constant with
respect to number of nodes. This change would require the introduction of a
worker thread whose job would be to send data to other nodes. This change is
the single largest opportunity for easy performance improvement.

While a node is joining, no changes can be made to the system as the joining
node is syncing with the others. A current limitation of the system is that it
does not block on modifying requests submitted during this time. Instead, it
returns the error EAGAIN ("Resource temporarily unavailable"). While this
behavior is safe, it may confuse applications that aren't prepared for that
possibility, and this approach does not guarantee that they will repeatedly try
their operation (causing unnecessary load on the system in the process).

While the system does support view changes (of nodes joining, dying and total
network partitions), the node joining algorithm relies heavily on \texttt{rsync}
to do all of the heavy lifting. This has many downsides, not the least of which
is that the two machines must be able to communicate with each other via a
seperate side channel that supports rsync (such as ssh in the current
implementation). To support efficient syncing on its own, the system could
employ the common method of logging so that the log could be replayed on a
remote server.

The system also does not support "partial" partitions at all (meaning when
the connection graph is connected, but not fully connected). To support this
possibility, AJFS would need to support a gossip-like communication mechanism.
It would also need to support sender-side operation queuing as described above.

Finally, the system does not provide protection against adversarial or malicious
systems. A future avenue of work would be to provide some level of protection
against malicious abuse of the RPC system or some form of byzantine fault
tolerance.
