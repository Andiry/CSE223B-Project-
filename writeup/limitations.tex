
The system suffers from a number of limitations that could be improved on in the
future. The greatest issue in AJFS at present is that operations are queued for
execution on the receiving server, rather than on the sending one. This was
chosen for simplicity, as AJFS is able to leverage the buffered socket
connection to queue requests. This approach requires, however, that every
request be immediately forwarded on to other hosts.

\subsection{Asynchronous Data Transmission}

Operations that need to be fast (for instance, \texttt{write()}) suffer from
this behavior, as data must  be fully transferred to other hosts before the
operation is able to return to the user. Even though the actual operation
execution is asynchronous, the data transmission often dominates the time it
takes to perform the operation, and that is fully synchronous.

This results in bandwidth that is inversely related to the number of nodes
that are active in the system. If AJFS queued outgoing requests and handled
transmitting of the data asynchronously, bandwidth would decoupled from the 
number of nodes in the system. This change would require the introduction of one
or more worker threads whose job would be to send data to other nodes. This
change is the single largest opportunity for easy performance improvement, and
would also reduce node death as described below.

\subsection{Synchronous RPC}

While working on AJFS, we discovered an issue with Thrift that results in
significant limitations for the system. When the system is under heavy load, a
TCP connection will fail between servers and Thrift will fail to successfully
transfer data to the receiving server, but will not throw an exception
identifying that the transfer was not successful.

Since the sending system believes that the operation succeeded, and the
operations are asynchronous and return nothing to the server, the sender has no
means of determining that the operation wasn't successful. Moreover, since a
server doesn't store operations after their sent, there's no way for the client
to subsequently request that operations are resent.

The same code additions in the client that allow the receiving server to ensure
idempotence allow us to detect these errors when they happen, but the only
options are to unsafely ignore them, or kill the server.

If the asynchronous data system described in the previous section was
implemented, it would be trivial to replace all of the thrift "oneway" functions
with synchronous RPC calls. This change would allow the sending system to verify
that the operation was received successfully.

\subsection{Blocking on Node Join}

While a node is joining, no changes can be made to the system as the joining
node is syncing with the others. A current limitation of the system is that it
does not block on modifying requests submitted during this time. Instead, it
returns the error EAGAIN ("Resource temporarily unavailable"). While this
behavior is safe, it may confuse applications that aren't prepared for that
possibility, and this approach does not guarantee that they will repeatedly try
their operation (causing unnecessary load on the system in the process).

\subsection{Syncing independent of \texttt{rsync}}

While the system does support view changes (of nodes joining, dying and total
network partitions), the node joining algorithm relies heavily on \texttt{rsync}
to do all of the heavy lifting. This has many downsides, not the least of which
is that the two machines must be able to communicate with each other via a
seperate side channel that supports rsync (such as ssh in the current
implementation). To support efficient syncing on its own, the system could
employ the common method of logging so that the log could be replayed on a
remote server.

\subsection{Changes in Design Decisions}

The system does not support "partial" partitions at all (meaning when the
connection graph is connected, but not fully connected). To support this
possibility, AJFS would need to support a gossip-like communication mechanism.
It would also need to support sender-side operation queuing as described above.

Finally, the system does not provide protection against adversarial or malicious
systems. A future avenue of work would be to provide some level of protection
against malicious abuse of the RPC system or some form of byzantine fault
tolerance.
