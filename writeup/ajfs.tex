%-----------------------------------------------------------------------------
%
%               Template for sigplanconf LaTeX Class
%
% Name:         sigplanconf-template.tex
%
% Purpose:      A template for sigplanconf.cls, which is a LaTeX 2e class
%               file for SIGPLAN conference proceedings.
%
% Author:       Paul C. Anagnostopoulos
%               Windfall Software
%               978 371-2316
%               paul@windfall.com
%
% Created:      15 February 2005
%
%-----------------------------------------------------------------------------

%\documentclass{sigplanconf}
\documentclass[preprint]{sig-alternate-10pt}
\renewcommand{\ttdefault}{lmtt}

% The following \documentclass options may be useful:
%
% 10pt          To set in 10-point type instead of 9-point.
% 11pt          To set in 11-point type instead of 9-point.
% authoryear    To obtain author/year citation style instead of numeric.

\usepackage{amsmath}
\usepackage[T1]{fontenc}
\usepackage[scaled=0.9]{helvet}
\usepackage{textcomp}
\makeatletter % *sigh* plus/minus isn't in textcomp, so we need it from LY1.
\input{ly1enc.def}
\makeatother
\DeclareTextSymbolDefault{\textplusminus}{LY1}
\usepackage{xspace}
\usepackage{multicol}
\usepackage{fancyvrb}
\usepackage{url}
\usepackage{graphicx}
\usepackage{color}
%\usepackage{setspace}
%\usepackage{flafter}
\usepackage{alltt}
\usepackage{floatflt}
\usepackage{subfigure}
\usepackage{microtype}\addtolength{\textwidth}{1.4pt}\addtolength{\oddsidemargin}{-1.4pt}\addtolength{\evensidemargin}{-1.4pt}


\newbox\subfigbox	% Create a box to hold the subfigure.
\makeatletter
\newenvironment{subfigenv}
{\def\caption##1{\gdef\subcapsave{\relax##1}}%
\let\reallabel\label
\def\label##1{\gdef\sublabsave{##1}}%
\let\subcapsave\@empty
\begin{lrbox}{\subfigbox}}
{\end{lrbox}
\subfigure[\subcapsave]{\usebox{\subfigbox}\reallabel{\sublabsave}}}
\makeatother

% \usepackage{setspace}
% \setstretch{0.96}
%\doublespacing

\raggedbottom

%!TEX root = ocm.tex

\definecolor{deletedtext}{rgb}{1,0.5,0.5}
\DeclareTextFontCommand{\textdel}{\color{deletedtext}}
\newenvironment{deleteme}{\color{deletedtext}}{}

\definecolor{annoteblue}{rgb}{0,0,1}
\DeclareTextFontCommand{\textnote}{\color{annoteblue}}
\newenvironment{notes}{\begin{quote}\itshape\color{annoteblue}}{\end{quote}}

\newif\ifanonymous
%\anonymoustrue

\newcommand{\name}[2]{\ifanonymous #1\else #2\fi} 

\newif\ifnotes
\notestrue

\newif\ifalt
\alttrue

\ifnotes
\newcommand{\cas}[1]{\textnote{\textbf{[CAS: #1]}}}
\newcommand{\meo}[1]{\textnote{\textbf{[MEO: #1]}}}
\newcommand{\jm}[1]{\textnote{\textbf{[JM: #1]}}}
\newcommand{\amc}[1]{\textnote{\textbf{[AMC: #1]}}}
\newcommand{\sap}[1]{\textnote{\textbf{[SAP: #1]}}}
\newcommand{\sab}[1]{\textnote{\textbf{[SAB: #1]}}}
\newcommand{\mjd}[1]{\textnote{\textbf{[JD: #1]}}}
\newcommand{\group}[1]{\textnote{\textbf{[GROUP: #1]}}}
\else
\newcommand{\cas}[1]{}
\newcommand{\meo}[1]{}
\newcommand{\jm}[1]{}
\newcommand{\amc}[1]{}
\newcommand{\sap}[1]{}
\newcommand{\sab}[1]{}
\newcommand{\mjd}[1]{}
\newcommand{\group}[1]{}
\fi


\let\cTX\texttt
\let\cTY\texttt
\let\cRW\texttt
\let\keyterm\emph
\let\prim\texttt %CAS

\newcommand{\atomic}{\cRW{atomic}\xspace}
\newcommand{\spawn}{\cRW{spawn}\xspace}
\newcommand{\yield}{\cRW{yield}\xspace}
\newcommand{\YIELD}{\texttt{\textbf{yield}}\xspace}
\newcommand{\yielding}{\cRW{yielding}\xspace}
\newcommand{\maybeyield}{\cRW{maybeYield}\xspace}
\newcommand{\blockuntil}{\cRW{blockUntil}\xspace}
\newcommand{\retry}{\cRW{retry}\xspace}
\newcommand{\orelse}{\cRW{orElse}\xspace}
\newcommand{\unprotected}{\cRW{unprotected}\xspace}
%\newcommand{\unsynchronized}{\cRW{unsynchronized}\xspace}
\newcommand{\yieldUntil}{\cRW{yieldUntil}\xspace}
\newcommand{\yielduntil}{\cRW{yieldUntil}\xspace}
\newcommand{\fork}{\cRW{fork}\xspace}
\newcommand{\synchronized}{\cRW{synchronized}\xspace}

\newcommand{\Pthreads}{\textrm{Pthreads}\xspace}
\newcommand{\Pth}{\textrm{Pth}\xspace}
%\newcommand{\Unix}{\textrm{Unix}\xspace}
\newcommand{\Posix}{\textrm{Posix}\xspace}
\newcommand{\Tinystm}{\textrm{TinySTM}\xspace}
\newcommand{\TLtwo}{TL2\xspace}
\let\Unix\Posix
\newcommand{\Java}{Java\xspace}
\newcommand{\Naive}{Na\"{\i}ve\xspace}
\newcommand{\naive}{na\"{\i}ve\xspace}

% Maybe *Observably* Cooperative Multithreading would be simpler?
\newcommand{\OCM}{Observationally Cooperative Multithreading\xspace}

\hyphenation{time-stamp}
\hyphenation{time-stamps}

%%% Local Variables: 
%%% mode: latex
%%% TeX-master: "ocm"
%%% End: 


\begin{document}


%%% 3)  You must start your paper with the \maketitle command.  Prior to the
%%%     \maketitle you must have \title and \author commands.  If you have a
%%%     \date command it will be ignored; no date appears on the paper, since
%%%     the proceedings will have a date on the front cover.


% \titlebanner{banner above paper title}        % These are ignored unless
% \preprintfooter{short description of paper}   % 'preprint' option specified.

\title{AJFS: A Simple Distributed File-System}

\ifanonymous
\author{[Author(s) Omitted --- Anonymized Submision]\\\
       [Institution(s) Omitted]\\\
       [Email address(es) Omitted]} %email

\else
\author{Joe DeBlasio, Jian Xu\\\
       University of California, San Diego \\\
       \{jdeblasio@cs, jix024@eng\}.ucsd.edu}
\fi

\maketitle

\begin{abstract}
    This is an abstract!
\end{abstract}

\section{Introduction}
I'm introduced!

\section{Motivation}
I'm motivated!

\section{Methods}
I'm methodical!

\section{Results}
I'm resultant!

\section{Conclusions}
I'm conclusive!

%\category{D.1.3}{Programming Techniques}{Concurrent
%  Programming---Parallel programming}
%\category{D.3.2}{Programming Languages}{Language
%  Classifications---Concurrent, distributed, and parallel languages}

%\terms
%Languages, Performance

%\keywords
%

%\input{intro.tex}

%\input{motivation.tex}

%\ifalt
%\input{background-alt.tex}
%\else
%\input{background.tex}
%\fi

%\input{oursolution.tex}

%\input{semantics.tex}

%We implement AJFS with C++, based on FUSE 2.9.2 "fusexmp\_fh" example and thrift. It consists of about 2800 lines of code, and can be divided into three major function components:

\emph{Host}: Host is the instance of thrift host, which provides remote access services to each other. The services consists of two categories, file operations and lock services. File operations include all the fuse file operations, except for the "read" type operations mentioned in Section \ref{sec:designOverview}. Lock operations include \texttt{lock()}, \texttt{join()}, \texttt{requestJoinLock()} and \texttt{getJoinLock()}. Section \ref{sec:designOverview} and \ref{sec:joinProcedure} have introduced the services.

\emph{Lock}: Lock consists of two categories: join lock and file lock. Join lock is for server joining and file lock is for file protection. When a server joins in a cluster, the join lock needs to be acquired before doing any file rsync work, this guarantees the file consistency after server joined in. File lock is used to maintain file consistency with multiple users. When a file is locked for writing, it cannot be written by another user. Unlike other cloud storage applications which provide a copy of the original file to allow simultaneously editing, we use lock to achieve safety and simple implementation.

\emph{File operations}: File operations in AJFS are inherited from FUSE, but we repack them to two components: local file operations and fuse operations. This is to simplify the seperation of file operation requests from local server and remote servers. A file operation request from local server will go into fuse operation handler, which will propagate the request to other servers as well as performing the local request. File operation requests from remote servers will go to thrift server handler, which will perform local file operations only.

When AJFS is executed, we launch three threads to handle the three components seperately. This helps to make the function components easy to understand and debug.


%\input{performance.tex}

%\input{debugging.tex}

%% TODO - this needs to be fleshed out.

We've presented AJFS, a fully replicated network file system designed for
loosely similar use cases to cloud storage systems like Dropbox.

While performance is not a primary objective in systems like this, the system
has significant performance limitations that prevent large-scale adoption.
Moreover, the lack of support for hosts without connectivity prohibits
true replacement of services like Dropbox.

In its current state, the system is a slightly better match for replicating NFS
across systems. None the less, AJFS successfully provides a safe, fully
replicated file system that supports view changes of several varieties.




%\ifalt
%\input{relatedwork-alt.tex}
%\else
%\input{relatedwork.tex}
%\fi

%\input{grant_proposal.tex}

%\appendix
%\section{Appendix Title}
%
%This is the text of the appendix, if you need one.


% We recommend abbrvnat bibliography style.

\bibliographystyle{abbrvnat}

% The bibliography should be embedded for final submission.

%\softraggedright

\bibliography{ajfs}

\end{document}
