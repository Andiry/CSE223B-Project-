%-----------------------------------------------------------------------------
%
%               Template for sigplanconf LaTeX Class
%
% Name:         sigplanconf-template.tex
%
% Purpose:      A template for sigplanconf.cls, which is a LaTeX 2e class
%               file for SIGPLAN conference proceedings.
%
% Author:       Paul C. Anagnostopoulos
%               Windfall Software
%               978 371-2316
%               paul@windfall.com
%
% Created:      15 February 2005
%
%-----------------------------------------------------------------------------

%\documentclass{sigplanconf}
\documentclass[preprint]{sig-alternate-10pt}
\renewcommand{\ttdefault}{lmtt}

\usepackage{amsmath}
\usepackage[T1]{fontenc}
\usepackage[scaled=0.9]{helvet}
\usepackage{textcomp}
\makeatletter % *sigh* plus/minus isn't in textcomp, so we need it from LY1.
\input{ly1enc.def}
\makeatother
\DeclareTextSymbolDefault{\textplusminus}{LY1}
\usepackage{xspace}
\usepackage{multicol}
\usepackage{fancyvrb}
\usepackage{url}
\usepackage{graphicx}
\usepackage{color}
%\usepackage{setspace}
%\usepackage{flafter}
\usepackage{alltt}
\usepackage{floatflt}
\usepackage{subfigure}
\usepackage{microtype}\addtolength{\textwidth}{1.4pt}\addtolength{\oddsidemargin}{-1.4pt}\addtolength{\evensidemargin}{-1.4pt}


\newbox\subfigbox	% Create a box to hold the subfigure.
\makeatletter
\newenvironment{subfigenv}
{\def\caption##1{\gdef\subcapsave{\relax##1}}%
\let\reallabel\label
\def\label##1{\gdef\sublabsave{##1}}%
\let\subcapsave\@empty
\begin{lrbox}{\subfigbox}}
{\end{lrbox}
\subfigure[\subcapsave]{\usebox{\subfigbox}\reallabel{\sublabsave}}}
\makeatother

% \usepackage{setspace}
% \setstretch{0.96}
%\doublespacing

\raggedbottom

%!TEX root = ocm.tex

\definecolor{deletedtext}{rgb}{1,0.5,0.5}
\DeclareTextFontCommand{\textdel}{\color{deletedtext}}
\newenvironment{deleteme}{\color{deletedtext}}{}

\definecolor{annoteblue}{rgb}{0,0,1}
\DeclareTextFontCommand{\textnote}{\color{annoteblue}}
\newenvironment{notes}{\begin{quote}\itshape\color{annoteblue}}{\end{quote}}

\newif\ifanonymous
%\anonymoustrue

\newcommand{\name}[2]{\ifanonymous #1\else #2\fi} 

\newif\ifnotes
\notestrue

\newif\ifalt
\alttrue

\ifnotes
\newcommand{\cas}[1]{\textnote{\textbf{[CAS: #1]}}}
\newcommand{\meo}[1]{\textnote{\textbf{[MEO: #1]}}}
\newcommand{\jm}[1]{\textnote{\textbf{[JM: #1]}}}
\newcommand{\amc}[1]{\textnote{\textbf{[AMC: #1]}}}
\newcommand{\sap}[1]{\textnote{\textbf{[SAP: #1]}}}
\newcommand{\sab}[1]{\textnote{\textbf{[SAB: #1]}}}
\newcommand{\mjd}[1]{\textnote{\textbf{[JD: #1]}}}
\newcommand{\group}[1]{\textnote{\textbf{[GROUP: #1]}}}
\else
\newcommand{\cas}[1]{}
\newcommand{\meo}[1]{}
\newcommand{\jm}[1]{}
\newcommand{\amc}[1]{}
\newcommand{\sap}[1]{}
\newcommand{\sab}[1]{}
\newcommand{\mjd}[1]{}
\newcommand{\group}[1]{}
\fi


\let\cTX\texttt
\let\cTY\texttt
\let\cRW\texttt
\let\keyterm\emph
\let\prim\texttt %CAS

\newcommand{\atomic}{\cRW{atomic}\xspace}
\newcommand{\spawn}{\cRW{spawn}\xspace}
\newcommand{\yield}{\cRW{yield}\xspace}
\newcommand{\YIELD}{\texttt{\textbf{yield}}\xspace}
\newcommand{\yielding}{\cRW{yielding}\xspace}
\newcommand{\maybeyield}{\cRW{maybeYield}\xspace}
\newcommand{\blockuntil}{\cRW{blockUntil}\xspace}
\newcommand{\retry}{\cRW{retry}\xspace}
\newcommand{\orelse}{\cRW{orElse}\xspace}
\newcommand{\unprotected}{\cRW{unprotected}\xspace}
%\newcommand{\unsynchronized}{\cRW{unsynchronized}\xspace}
\newcommand{\yieldUntil}{\cRW{yieldUntil}\xspace}
\newcommand{\yielduntil}{\cRW{yieldUntil}\xspace}
\newcommand{\fork}{\cRW{fork}\xspace}
\newcommand{\synchronized}{\cRW{synchronized}\xspace}

\newcommand{\Pthreads}{\textrm{Pthreads}\xspace}
\newcommand{\Pth}{\textrm{Pth}\xspace}
%\newcommand{\Unix}{\textrm{Unix}\xspace}
\newcommand{\Posix}{\textrm{Posix}\xspace}
\newcommand{\Tinystm}{\textrm{TinySTM}\xspace}
\newcommand{\TLtwo}{TL2\xspace}
\let\Unix\Posix
\newcommand{\Java}{Java\xspace}
\newcommand{\Naive}{Na\"{\i}ve\xspace}
\newcommand{\naive}{na\"{\i}ve\xspace}

% Maybe *Observably* Cooperative Multithreading would be simpler?
\newcommand{\OCM}{Observationally Cooperative Multithreading\xspace}

\hyphenation{time-stamp}
\hyphenation{time-stamps}

%%% Local Variables: 
%%% mode: latex
%%% TeX-master: "ocm"
%%% End: 


\begin{document}


%%% 3)  You must start your paper with the \maketitle command.  Prior to the
%%%     \maketitle you must have \title and \author commands.  If you have a
%%%     \date command it will be ignored; no date appears on the paper, since
%%%     the proceedings will have a date on the front cover.


% \titlebanner{banner above paper title}        % These are ignored unless
% \preprintfooter{short description of paper}   % 'preprint' option specified.

\title{AJFS: A Simple Distributed File-System}

\author{Joe DeBlasio, Jian Xu\\\
       University of California, San Diego \\\
       \{jdeblasio, jix024\}@cs.ucsd.edu}

\maketitle

\begin{abstract}
    Today, files are required to be accessable at multiple locations,
    and they need to be synchronized. It is common for people to
    store their files in a cloud storage, so they can continue work when they
    get home. Various cloud storage services are available today: Dropbox,
    Google Drive, SkyDrive, Apple iCloud, etc. For cloud storage, the most
    important thing is the consistency of the files.

    In this paper, We present AJFS: a fully-replicated distributed file system supporting
    multiple users, simultaneous editing of distinct files and robustness
    against nodes failing or joining the system. Designed to similar use cases
    as Dropbox, the system has been designed for simplicity, supporting causal,
    eventual consistency.
\end{abstract}

\section{Introduction}
\label{sec:introduction}

In today's world, people use a variety of computers from a variety of locations.
They demand that their files remain synchronized and accessible from wherever
they are. In this space, the winners have been cloud storage services such as
Dropbox~\cite{Dropbox}, Google Drive~\cite{googledrive}, Microsoft
SkyDrive~\cite{skydrive}, Apple iCloud~\cite{icloud} and others.

These services meet the need fairly effectively. As long as the user has
internet access, files remain available and largely synchronized. However, these
benefits are not without cost. A user must trust the cloud services to keep
their data secure and private, and synchronizing their files is subject to the
potentially relatively low bandwidth links between the user and the cloud
service.

We present AJFS: a fully-replicated file system that supports multiple users and
limited simultaneous editing across hosts. AJFS is robust against nodes failing
and joining the system, and is designed with a use case similar to Dropbox.

AJFS is not meant to replace Dropbox, but is instead meant as a first step
towards facilitating users to share files under similar usage patterns as
Dropbox without having to trust a third party service provided. While it does
provide a "shared folder" similar to that of Dropbox, it provides no support for
systems that are offline or unable to access all other nodes participating.

AJFS is fully functional and presently emphasizes data safety and correctness
over performance, but future work could dramatically decrease the performance
gap between local file systems and the AJFS system.



%TODO
We assume that other nodes are non-adversarial, and accept up to $n$ failures in $2n+1$ nodes.
%TODO

\section{Motivation}
\label{sec:motivation}
\input{motivation.tex}

\section{System Design}
\label{sec:designOverview}


AJFS is designed to be, above all else, simple both in its guarantees and in its
implementation. As such, several assumptions were made to keep the scope of the
project manageable. These assumptions allow the design methodology that we used
to be possible, and thus warrant discussion outright.

\subsection{Use Cases and Assumptions}

AJFS is aimed at a use case similar to Dropbox: while multiple users could be
editing files on multiple systems simultaneously, it is primarily designed for a
single user across multiple systems. The emphasis is on data availability across
systems, and is not designed with 'collaborative' uses in mind.

From a network perspective, AJFS assumes that, excepting in the case of a
partition, the nodes in the system will be fully connected (meaning all nodes
can communicate with all others). While a partition is a case we handle, we do
NOT handle a 'partial partition', wherein, for instance, node A can talk to node
B, and B and talk to C, but A is unable to communicate with A.

We also assume that communications between two nodes will be largely reliable.
While AJFS does allow some packet loss (as we use TCP connections, and try to
reestablish a connection when a connection fails), if the system is ultimately
unable to get a packet across, that node MUST die. AJFS nodes do not keep a
log of operations to execute.

Finally, for the sake of simplicity, we assume that nodes are non-adversarial,
and will never provide inaccurate information when they provide information at
all.

\subsection{Overview}

AJFS uses a very approach to a networked file system: every node in the system
contains a complete copy of the data. In a system with $2n + 1$ nodes, we can
accept up to $n$ failures. There are no 'primary' hosts (although for a node
joining, one node has a temporary special role as a facilitator).

All "read"-type operations (\texttt{access()}, \texttt{readdir()},\\ 
\texttt{getattr()}, \texttt{read()}) are serviced immediately by the local
system. All operations that modify the file system are first executed locally,
and then forwarded on to other servers.

Remote operations are queued on the receiving server, and are serviced in a FIFO
fashion. This guarantees a "partial causal order"-- requests originating
from the same host will happen in their original order, and thus causally.
Requests originating from different hosts have no such guarantees.

This is because it is possible for two nodes (say hosts B and C) to receive an
operation (from host A), have host B perform the operation, and generate a new
operation (caused by the first operation) that is queued at host C. Host C may
process host B's operation before the original operation from host A depending
on the scheduling policy.

As the intention of the system is to support multiple users only insofar as
they're working on disjoint sets of files (and thus their relative orderings
aren't important), we determined that this above limitation is acceptable.
Moreover, to protect against any 'accidental' collaboration that might cause bad
interference, files and file hierarchies are locked when in use. This prevents
one host from interfering with another, as before it makes any changes it must
get a lock across all live nodes in the system.

Operations succeed if they succeed locally. AJFS does not commit operations
separately from transmitting them. However, if at any time a node is unable to
communicate with another node, the node marks the incommunicado host as "dead".
If the node is able to communicate with the dead host in the future, the system
informs that host that they are out of data, and must rejoin via the join
procedure. Similarly, if a node determines that it is no longer in a quorum,
the node will exit.

When a new node wants to join the group, it communicates with one other node in
the system. This node handles all of the logistics of adding the node. This
process is described in Section~\ref{sec:joinProcedure}.

For simplicity, we define a "quorum" as a majority of the nodes that we've ever
seen alive. If we want to reduce the number of nodes in the system without
causing the system to fail, we must restart the system and only introduce that
number of nodes.



%\section{Command Queuing}
%    \input{command.tex}

\subsection{Join Procedure}
\label{sec:joinProcedure}

Joining the AJFS system requires knowing the hostname and port of at least one
active node in the system. Joining is a two-phase process. For convenience, we
denote the new node wishing to join the group the "joiner", and the active node
that the joiner knows about is the "facilitator".

Figure \ref{fig:algorithm} describes the join procedure of AJFS server. The joiner starts off by sending a \texttt{requestJoinLock()} RPC request to the
facilitator. The facilitator then ensures that no files are open for writing,
and then attempts to lock all of the hosts in the AJFS system using the
\texttt{getJoinLock()} RPC. If the facilitator is not able to get the "join
lock" from all other hosts, or if files are open for writing, it will return to
the joiner that it failed, and the joiner must try again at a later time.

If the facilitator is able to get a join lock from every other node in the
system, it returns back to the joiner that the request succeeded as well as the
absolute path on the facilitator for the facilitator's local database.

At this point, the joiner \texttt{rsync}s its own local database with the
database on the facilitator. \texttt{rsync} was chosen because it is fairly
efficient, is already available, and allows a brand new host that has never
joined before to join using the same procedure as a node that has previously
failed and is coming back.

When the joiner has finished \texttt{rsync}ing, it must start up all necessary
worker threads, and then send a \texttt{join()} RPC to the facilitator. The
facilitator in turn calls \texttt{addServer()} on all of the other nodes,
passing to them the information needed to contact the joiner. Afterwards, it
calls \texttt{releaseJoinLock()} on all of the servers. Finally, it then returns
to the new joiner the list of other hosts that the facilitator knows are in the
system.

In order for this to be safe, several properties have to hold. When a node 
holds the "join lock", no additional modifications to the file system can
proceed. As a result, local operations must block until the "join lock" is
released. Similarly, the joiner must not make any changes to its own file system
until after \texttt{join()} returns.

To be robust to failures of both the joiner and the facilitator, the "join
locks" must be able to timeout. This process is described in
Section~\ref{sec:hostManager}. A node is fully "alive" only once
\texttt{releaseJoinLock()} has been called on \textbf{all} other live servers.
If any call fails, the joining server will eventually been told that it must
rejoin by the node whose \texttt{releaseJoinLock()} was not called.

\begin{figure}[Ht]
\includegraphics[width=\linewidth]{algorithm.pdf}
\caption{Pseudo-code for AJFS server join}
\label{fig:algorithm}
\vspace{-5mm}
\end{figure}

%\begin{algorithm}
%\caption{\figtitle{Pseudo-code for AJFS join procedure}}
%\label{algo:joinProcedure} 

%\begin{algorithmic}

%\Function{A.ServerStart with B as remote host}{\null}
%\State A calls B.requestJoinLock()
%	\For{Server X in B's server list}
%		\State B calls X.getJoinLock()
%		\If{$fail\ to\ get\ join\ lock$}
%			\For{Server Y in B's lock grabbed list}
%				\State B calls Y.releaseJoinLock()
%				\State return failure
%		\EndIf
%	\EndFor
%\State A rsync with B
%\State A calls B.join()
%	\For{Server X in B's server list}
%		\State B calls X.addServer(A)
%	\EndFor
%	\For{Server X in B's server list}
%		\State B calls X.releaseJoinLock()
%	\EndFor
%	\State B returns server list to A
%\State A update server list
%\EndFunction
%\end{algorithmic}
%\end{algorithm} 


\subsection{File Locking}
\label{sec:fileLocking}
As our file system does not support a total causal ordering between hosts on
operations, some additional work is required to prevent breaking the minimal
causal ordering required by many file operations.

The system does permit multiple users to operate simultaneously, but they must
be working on a disjoint set of files/directories in order to maintain safety.
To enforce this, when a file or folder is opened, it is first synchronously
locked across all hosts. If a lock can not be acquired for any host, the
operation will fail (and all nodes will be notified of such). When a file/folder
is closed, it is unlocked.

AJFS uses a synchronized set of multiple-reader-exclusive-writer locks for both
files and directories across the hosts of the system. Since files do not exist
in a vacuum but instead exists within its parent directory, in order to get a
lock on a file, AJFS must also acquire a read lock on all directories above the
file in the hierarchy.

For example, if we wanted to open a file \texttt{/a/b/c} for writing, we would
have to acquire an exclusive write lock on the file \texttt{/a/b/c}, as well as
read locks on \texttt{/a/b/}, \texttt{/a/} and \texttt{/}. This allows us to
safely modify \texttt{a} without worrying that another host will edit the file
or a parent directory and risk giving inconsistent ordering guarantees across
hosts.

Because local file system semantics don't enforce a locking constraint like
this, and because ordering is deterministic and causal when it originates from
a single host, AJFS allow multiple writers, or simultaneous readers and writers,
so long as they're all on the same system. This may yield bizarre results, but
they will be consistent results across the nodes in the system.


\subsection{Host Manager}
\label{sec:hostManager}
The Host Manager is a dedicated thread and is responsible for maintaining the
connections to all other known hosts. Every few seconds (currently configured to
2.5 seconds), the Host Manager sends an asynchronous RPC \texttt{ping()} to each
of the hosts believed to be alive. This \texttt{ping()} is designed to be
minimally intrusive, doing nothing more than returning immediately.

Even though these requests are "asynchronous" (Thrift's "oneway"), requests are
actually queued on the receiving server when they're initially sent. In this
way they're synchronously transmitted, but asynchronously executed. As a
result, we still must be able to successfully communicate with the remote
system immediately. If the request fails, that server is marked as dead, and
any file locks that the remote host holds on the local machine are forceably
released and no further communication will be attempted with that remote host
unless it goes back through the join procedure.

By keeping track of what hosts succeed or fail, the Host Manager also verifies 
that it is still in the majority of hosts. If it is not, the Host Manager knows
that either it has been partitioned into a minority partition, or that more
than half of the hosts in the group are down. In either of these cases, the
system exits.

The Host Manager is also responsible for timeouts on the "join lock". The
system maintains a configurable maximum allowed time period (currently 10
seconds) that a joining server can hold a "join lock" on the file system. This
corresponds to roughly the time between calls by a new server to
\texttt{requestJoinLock()} and \texttt{join()} on an already-joined server. If
the timeout expires before the "join lock" is released, we mark the new server
as dead (and refuse to respond to any subsequent RPC calls without
starting the join procedure over) and unlock the system.

Each server is responsible for resetting their own "join lock" (in case the
system that failed was the server facilitating the join). However, the
facilitating server has a "join lock" timeout slightly earlier than the others
to help prevent the possibility of a timeout occurring on one server, but having
\texttt{join()} start on the facilitator before the "join lock" times out there.
This case is safe, as \texttt{join()} will fail, but is undesirable as an
join procedure that would otherwise proceed must start again.

This period corresponds to the time given for the call to \texttt{rsync} to
complete. It is entirely possible that \texttt{rsync} may need more time to
sync the two systems, but we leave a better solution to future work.
It could be fixed by having a 'heartbeat' from the joining server sent
periodically to reset the timer. Our solution is still safe, and subsequently
likely to succeed, as eventually \texttt{rsync} will return. At this point, the
joining server will attempt to \texttt{join()} with the facilitating server.
This call will fail, but as \texttt{rsync} has just recently completed,
subsequent attempts to join will likely result in rsync completely more
quickly.




\subsection{File System Support}
\label{sec:fsSupport}

Our file system support is built on top of FUSE. FUSE allows us to implement
a full-fledged filesystem that can be mounted by a host OS just like any other
file system. Moreover, it allows all code to be written in userspace.

Our file system is based on a provided example that's included with FUSE:
"fusexmp\_fh". This example simply mirrors the root of the host's file system
into the FUSE filesystem (i.e. such that \texttt{~/mountPt/tmp/} is equivalent to
\texttt{/tmp/}).

We modified this base filesystem to support mirroring a non-root directory. In
this way, our file system mirrors a local directory that ideally would be hidden
from the end user. We additionally seperated operations to the local
(non-mounted) file system and the operations initiated by operations on the
mounted filesystem (the FUSE operations) such that we could add our own remote
code without interfering with the local operations. Finally, we added in support
for a few operations that the example omitted (for instance, support for
updating access and modification times).

AJFS uses the local mirroring system as a black-box database for AJFS.
Operations on the mounted filesystem are completed on the local file system, and
when necessary, remote calls to other servers at the same time. Thus,
operations on the local file system can be initiated from either of two
locations: from a local call (initiated by a FUSE callback) or by remote call
(initiated by a Thrift RPC call).

We dedicate one worker thread to handle FUSE-initiated operations on the local
mounted filesystem.



\subsection{RPC Service}
\label{sec:rpcService}
RPC support is handled by a dedicated thread running an Apache Thrift Non-blocking
server. Thrift was chosen for convenience and ease of getting started (due to
our familiarity with it).

We use a new RPC for each \textit{modifying} file system call (i.e.
\texttt{write()}), as well as for opening/closing, locking/unlocking, pinging,
the join procedure. All procedures use Thrift's "oneway" designation, except
where synchronous behavior is required (such as in \texttt{lock()},
\texttt{fsync()} and join procedure RPCs).

Thrift's "oneway" modifier simply means that the client doesn't wait for the
server to finish the operation before returning (and thus the RPC must be
\texttt{void}). Normally, Thrift does not guarantee that multiple "oneway" RPCs
are executed serially or in the proper order. This is because in a
multi-threaded Thrift server, the server may schedule RPCs to different worker
threads.

In addition to the normal arguments, most operations also require a host
operation counter. This counter allows remote operations to be idempotent. When
the Thrift server receives an RPC from a sending host with a counter value less
than or equal to that of a previous call, the server drops the operation instead
of acting on the request.

Since we must provide a causal ordering for operations from the same host, we
use a single-threaded version of Thrift's Non-blocking server. Since it can only
service a single request at a time, and since it does not (usually) interrupt an
execution of an RPC to service a new one, this server maintains the ordering
requirements that we need.

We use Thrift's Non-blocking server instead of Thrift's "SimpleServer". This is
because "SimpleServer" is only able to service a single connection at a time,
and will not go onto a new server until that connection is closed. As an
optimization, we chose to keep connections between the servers open between RPC
calls. This avoids the added overhead of initializing and closing a new TCP
connection for every RPC, but necessitated a new server in order to support
more than two nodes.



\section{Evaluation}
\label{sec:evaluation}
We test AJFS on a local machine as well as sysnet cluster. The evaluation includes consistency part and performance part.

\subsection{Consistency}

The main design purpose of AJFS is the file consistency when accessed by multiple servers. We've tested the following cases:

\subsubsection{Server crash and rejoin}

In distributed system, it's common that server crashes. We need to guarantee that when a server is down, the data is not corrupted and when the server is back online, the data will be synchronized.

The design of AJFS supports server rejoin. Since it's running based on the local backup, client can access the file locally without issue, just like Coda~\cite{KS92} does.
The only problem is when the server is down, client cannot access the mount point, because fuse server is also down. Instead, client should access the local backup. When the server rejoins, it will automatiaclly rsync with remote repositories and sync up the local files with remote replicas.

We tested following test cases on AJFS:

1. Start three AJFS servers A, B and C; kill server C; client a on server A creates file x; restart server C.

\emph{Result}: client c on Server C can get file x automatically.

The test proves AJFS's robustness: server can join and crash randomly, but file consistency is kept among replicas.

\subsubsection{Write protection}

Many distributed systems support mulitple simultaneous writes. TreadMarks~\cite{treadmarks} supports multiple writers; Bayou~\cite{bayou} askes each writer to provide dependency check and merge procedures.
AJFS does not support multiple simultaneous writers though, this is mainly for simplicity reason. 

A possible solution to support multiple simultaneous writers in AJFS is copy on write. The write operation only writes to the local copy, and when the file is closed, the modifications is propagated to the replicas. We do not use this mechanism because the merge procedure on server the side may be complicated. Another solution to this issue is to keep different conflict versions of a single file. 
However, we choose locks and only allow one writer to a file at one time, which is simple and the correctness is guaranteed.

We tested following test cases on AJFS:

1. Client a on server A writes to file x; at the same time client b on sever B opens file x for read.

2. Client a on server A writes to file x; at the same time client b on sever B opens file x for write.

3. Client a on server A read file x; at the same time client b on sever B deletes the folder which contains file x.

On AJFS, all the test cases work as expected; For test case 1, client b can open the file for read, because we allow one writer and multiple readers simultaneously. For test case 2, the request from client b is declined. For test case 3, AJFS denies the operation, specifies the folder is not empty. Hence we guarantee the file consistency on AJFS.

\subsection{Performance}

Performance is not the key motivation of AJFS. Use AJFS to sync up big files is not efficient, this is because AJFS needs to propagate the data to all the nodes during the write.
However, AJFS supports big file sync up.

Figure \ref{fig:writeperf} shows the performance of AJFS write, which is measured using dd.
There is a clear trend that with the number of AJFS nodes increasing,
the write bandwidth of AJFS decreases. This is because although AJFS uses asynchronous write propagation, that it does not wait for the response from remote nodes,
transfer the data itself adds additional overhead. 
More nodes means more data needs to transfer, hence the performance degrades with the number of nodes increasing.
That's why AJFS does not suit for big files sync up.

We can optimize the write performance by using additional thread, which queues up write requests and propagate them to remote nodes asynchronously.
This will improve the write performance of AJFS and makes the write performance constant.

Figure \ref{fig:readperf} shows the read performance of AJFS. It is constant with number of nodes increasing, and is also close to Fuse read performance.
This is because AJFS handles read locally and does not propagate non-modification request.
%However, we compare the AJFS system with a local filesystem. The test is run by bonnie++ 1.03e~\cite{bonnie++}. Table \ref{table:performance} shows the AJFS bandwidth performance. 

As we can see, since FUSE is a user space file system, it adds additional
 overhead to the file operations and hence the performance is worse than the local file system.
Also, the write performance of AJFS is lower than FUSE with number of nodes increasing, as it needs to propagate the requests to all the remote nodes.
 This is a limitation with current AJFS implementation and can be resolved by propagating write requests asynchronously.

\begin{figure}[Ht]
\includegraphics[width=\linewidth]{readperf.pdf}
\caption{AJFS Read Performance}
\label{fig:readperf}
\vspace{-5mm}
\end{figure}

\begin{figure}[Ht]
\includegraphics[width=\linewidth]{writeperf.pdf}
\caption{AJFS Write Performance}
\label{fig:writeperf}
\vspace{-5mm}
\end{figure}

%\begin{table}[Ht]
%\caption{AJFS read/write performance (KB/s)}
%\centering
%\begin{tabular}{|p{0.9cm}|p{0.9cm}|p{0.9cm}|p{0.9cm}|p{0.9cm}|p{0.9cm}|}
%\hline\hline
%System & Byte Write & Block Write & Rewrite & Byte Read & Block Read \\
%heading
%\hline
%Ext3	& 49090	& 55403	& 17766	& 45094	& 66464	\\
%\hline
%FUSE	& 33792	& 47490	& 15417	& 39853	& 78666	\\
%\hline
%AJFS	& 13972	& 19651	& 11383	& 41987	& 74125	\\
%\hline
%\end{tabular}
%\label{table:performance}
%\end{table}


\section{Limitations}
\label{sec:limitations}

The system suffers from a number of limitations that could be improved on in the
future. The greatest issue in AJFS at present is that operations are queued for
execution on the receiving server, rather than on the sending one. This was
chosen for simplicity, as AJFS is able to leverage the buffered socket
connection to queue requests. This approach requires, however, that every
request be immediately forwarded on to other hosts.

\subsection{Asynchronous Data Transmission}

Operations that need to be fast (for instance, \texttt{write()}) suffer from
this behavior, as data must  be fully transferred to other hosts before the
operation is able to return to the user. Even though the actual operation
execution is asynchronous, the data transmission often dominates the time it
takes to perform the operation, and that is fully synchronous.

This results in bandwidth that is inversely related to the number of nodes
that are active in the system. If AJFS queued outgoing requests and handled
transmitting of the data asynchronously, bandwidth would decoupled from the 
number of nodes in the system. This change would require the introduction of one
or more worker threads whose job would be to send data to other nodes. This
change is the single largest opportunity for easy performance improvement, and
would also reduce node death as described below.

\subsection{Synchronous RPC}

While working on AJFS, we discovered an issue with Thrift that results in
significant limitations for the system. When the system is under heavy load, a
TCP connection will fail between servers and Thrift will fail to successfully
transfer data to the receiving server, but will not throw an exception
identifying that the transfer was not successful.

Since the sending system believes that the operation succeeded, and the
operations are asynchronous and return nothing to the server, the sender has no
means of determining that the operation wasn't successful. Moreover, since a
server doesn't store operations after their sent, there's no way for the client
to subsequently request that operations are resent.

The same code additions in the client that allow the receiving server to ensure
idempotence allow us to detect these errors when they happen, but the only
options are to unsafely ignore them, or kill the server.

If the asynchronous data system described in the previous section was
implemented, it would be trivial to replace all of the thrift "oneway" functions
with synchronous RPC calls. This change would allow the sending system to verify
that the operation was received successfully.

\subsection{Blocking on Node Join}

While a node is joining, no changes can be made to the system as the joining
node is syncing with the others. A current limitation of the system is that it
does not block on modifying requests submitted during this time. Instead, it
returns the error EAGAIN ("Resource temporarily unavailable"). While this
behavior is safe, it may confuse applications that aren't prepared for that
possibility, and this approach does not guarantee that they will repeatedly try
their operation (causing unnecessary load on the system in the process).

\subsection{Syncing independent of \texttt{rsync}}

While the system does support view changes (of nodes joining, dying and total
network partitions), the node joining algorithm relies heavily on \texttt{rsync}
to do all of the heavy lifting. This has many downsides, not the least of which
is that the two machines must be able to communicate with each other via a
seperate side channel that supports rsync (such as ssh in the current
implementation). To support efficient syncing on its own, the system could
employ the common method of logging so that the log could be replayed on a
remote server.

\subsection{Changes in Design Decisions}

The system does not support "partial" partitions at all (meaning when the
connection graph is connected, but not fully connected). To support this
possibility, AJFS would need to support a gossip-like communication mechanism.
It would also need to support sender-side operation queuing as described above.

Finally, the system does not provide protection against adversarial or malicious
systems. A future avenue of work would be to provide some level of protection
against malicious abuse of the RPC system or some form of byzantine fault
tolerance.


\section{Related Work}
\label{sec:relatedWork}

AJFS takes great inspiration from other distributed systems. Originally designed
to be very similar to Harp \cite{LGG91}, most of the logging characteristics
were eventually entirely stripped out as being unnecessary.

AJFS uses a similar manager process to that as described in Fragipani from
\cite{TML97}. Their lock manager is tasked with periodically checking for
liveness on the locks and hosts, and when it detected the death of a server,
their lock manager would pass the lockset and log from the dead server onto a
live server.

In Fragipani, the lock manager was a single service running on a single
machine. The lock manager was responsible for maintaining the lock set as well
as checking for liveness as described previously.

In AJFS, our host manager is similarly responsible for checking for liveness of
other servers. However, when a dead server is detected, its locks are simply
released. Moreover, the lock manager is a process that runs on every node in the
system (as no node is special), and is uniquely responsible for managing the
local copy of the lockset.

AJFS uses a conceptually similar operation queuing mechanism to that of Coda
\cite{KS92}. In Coda, client reads are serviced by the local preferred server,
but client's writes are stamped with a local server ID, and then forwarded on to
other servers.

AJFS is a much simpler system than Coda is, and AJFS doesn't ever use timestamps
of any variety. Moreover, AJFS is a fully-distributed networked file system,
while Coda is a true network-based, distributed file system. Coda's able to
ensure full causal ordering, even amid nodes leaving and rejoining. AJFS uses
none of the complexity that Coda does.


\section{Conclusions}
% TODO - this needs to be fleshed out.

We've presented AJFS, a fully replicated network file system designed for
loosely similar use cases to cloud storage systems like Dropbox.

While performance is not a primary objective in systems like this, the system
has significant performance limitations that prevent large-scale adoption.
Moreover, the lack of support for hosts without connectivity prohibits
true replacement of services like Dropbox.

In its current state, the system is a slightly better match for replicating NFS
across systems. None the less, AJFS successfully provides a safe, fully
replicated file system that supports view changes of several varieties.




\section{Earlier Writeup}
Our project is to implement a fully-replicated distributed file system that
supports simple view changes, multiple users and simultaneous editing of files.
To keep the project manageable, a number of simplifying assumptions have been
put in place that limit the use cases under which the system will be used. Our
primary use case is similar to that of Dropbox-- sharing will primarily be for a
single user, and while files will need to be editable globally, they will not
likely be edited in multiple locations in rapid succession.

Because we are implementing a file system, we must support causal orderings of
operations from each individual host to the others. However, since we expect
that there will either only be a single user on a single host at a time, or that
multiple hosts will be editing disjoint sets of files, the system will not
support a total ordering of file system changes between hosts. This use case
also allows us to support a lazier ‘eventual’ consistency model in which all
hosts they will eventually see all of the changes to the file system.

To help ensure that hosts edit disjoint sets of files, file trees will be
synchronously locked with multiple-reader-single-writer locks when files are
opened for editing. This will permit users on other hosts to see changes as they
happen, but not interfere with those changes (as doing so would require
acquiring an exclusive lock).

Operations originating on host “A” will be queued locally and also broadcast to
be queued on each of the other hosts in a queue specifically for A’s requests.
The system will handle the requests in order. Requests will ‘succeed’ once a
majority of hosts have acknowledged that the instruction was queued (for
asynchronous requests) or the majority of hosts have processed the queue up
until the new operation (for synchronous requests)..

The system is designed to be fairly simple, and as a result, we’ve chosen to
handle view changes in a very simple way: nodes are ‘lost’ when they can’t
contact the majority of hosts. We assume that packet transmission is fairly
reliable, and that the network won’t have partial or frequent outages. If nodes
detect that they’re unable to connect to the majority of hosts, it is up to them
to ensure that they stop accepting new file system modifications, and attempt to
reconnect (or simply restart). Other nodes can also inform a node that that node
‘died’ (because it previously wasn’t responding) at which point, it will have to
restart.

When nodes rejoin the system, they request a global lock on the file system. At
this point, nodes will temporarily not permit writes (either by blocking on
write or by having writes fail), they will sync to their disks, and then the
joining node will synchronize with another node using rsync. Upon completion (or
after a timeout, if the system fails), the lock is released and the system
resumes. Because of the choice to use rsync, nodes must be able to reach each
other via an rsync-compatible method (like ssh) as well as via our own RPC
calls.

For simplicity, we are using FUSE to implement the local filesystem, and Thrift
for IPC. The use of Thrift effectively guarantees that this system will not be
likely to achieve very good performance. Given our consistency model, this is an
acceptable tradeoff. Moreover, thrift uses stubs that would be fairly easy to
replace with a higher performance IPC method down the line if we were interested
in extending the system. To further keep the system and logic simple, the system
will not support linking (either hard or symbolic), FIFOs, nor extended
attributes. Depending on how the system progresses, we may opt to remove
non-essential file mode permissions as well. Given the tools used and our
intended testing plan, we will only guarantee that our software artifact works
on installations of Ubuntu  12.04 with FUSE and Thrift installed.

Currently, the system is at a fairly young stage. We have working ‘local’ file
systems using FUSE. We can also communicate between instances of FUSE via
Thrift, but operations are not executed remotely. We encountered more trouble
than we care to admit getting FUSE and Thrift to coexist, and so that
represented a significant time sink.

Because of our choice of IPC, we will not be measuring success with substantial
benchmarking. We plan on developing a small test suite that will test
functionality of the system across 3-5 nodes. This will consist of nodes dying
and rejoining, and moving files into and out of the file system and checking
accuracy of those files (likely with hashes).

Given our host needs, we plan on just using the nodes that have already been set
up for class use (i.e. the systems used for labs 1-3). Given that these nodes
already have Thrift and FUSE installed, we don’t need anything additional.



%\category{D.1.3}{Programming Techniques}{Concurrent
%  Programming---Parallel programming}
%\category{D.3.2}{Programming Languages}{Language
%  Classifications---Concurrent, distributed, and parallel languages}

%\terms
%Languages, Performance

%\keywords
%


%\appendix
%\section{Appendix Title}
%
%This is the text of the appendix, if you need one.


% We recommend abbrvnat bibliography style.

\bibliographystyle{abbrvnat}

% The bibliography should be embedded for final submission.

%\softraggedright

\bibliography{ajfs}

\end{document}
