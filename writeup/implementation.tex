AJFS is written in C++ and uses FUSE 2.9.2 and Apache Thrift.
Figure~\ref{fig:architecture} shows the general architecture of AJFS. It
consists of about 2800 lines of code, and can be divided into three major
function components:

\emph{Thrift Server}: The Thrift server listens for RPC calls from other
instances of AJFS, providing operations for file manipulation, file locking and
adding new servers to an AJFS system. The file manipulation operations include
all standard modifying file operations. No "read" operations are included.
See Section~\ref{sec:designOverview} and \ref{sec:joinProcedure} for more
information about this functionality.

\emph{Host Manager}: The host manager maintains the connections to the other
servers in the AJFS system, and upon detecting a node's death, is in charge of
releasing both file locks and the join lock.

\emph{FUSE Server}: The FUSE server handles file operations originating from the
local system. The FUSE server is responsible for forwarding on calls to
perform local file operations, as well as forwarding on those calls to the other
servers in the system.

When AJFS is executed, we launch three threads to handle the three components
seperately. This helps to make the function components easy to understand and
debug.

