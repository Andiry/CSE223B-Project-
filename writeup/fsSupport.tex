
Our file system support is built on top of FUSE. FUSE allows us to implement
a full-fledged filesystem that can be mounted by a host OS just like any other
file system. Moreover, it allows all code to be written in userspace.

Our file system is based on a provided example that's included with FUSE:
"fusexmp\_fh". This example simply mirrors the root of the host's file system
into the FUSE filesystem (i.e. such that \texttt{~/mountPt/tmp/} is equivalent to
\texttt{/tmp/}).

We modified this base filesystem to support mirroring a non-root directory. In
this way, our file system mirrors a local directory that ideally would be hidden
from the end user. We additionally seperated operations to the local
(non-mounted) file system and the operations initiated by operations on the
mounted filesystem (the FUSE operations) such that we could add our own remote
code without interfering with the local operations. Finally, we added in support
for a few operations that the example omitted (for instance, support for
updating access and modification times).

AJFS uses the local mirroring system as a black-box database for AJFS.
Operations on the mounted filesystem are completed on the local file system, and
when necessary, remote calls to other servers at the same time. Thus,
operations on the local file system can be initiated from either of two
locations: from a local call (initiated by a FUSE callback) or by remote call
(initiated by a Thrift RPC call).

We dedicate one worker thread to handle FUSE-initiated operations on the local
mounted filesystem.

