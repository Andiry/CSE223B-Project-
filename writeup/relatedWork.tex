
AJFS takes great inspiration from other distributed systems. Originally designed
to be very similar to Harp \cite{LGG91}, most of the logging characteristics
were eventually entirely stripped out as being unnecessary.

AJFS uses a similar manager process to that as described in Fragipani from
\cite{TML97}. Their lock manager is tasked with periodically checking for
liveness on the locks and hosts, and when it detected the death of a server,
their lock manager would pass the lockset and log from the dead server onto a
live server.

In Fragipani, the lock manager was a single service running on a single
machine. The lock manager was responsible for maintaining the lock set as well
as checking for liveness as described previously.

In AJFS, our host manager is similarly responsible for checking for liveness of
other servers. However, when a dead server is detected, its locks are simply
released. Moreover, the lock manager is a process that runs on every node in the
system (as no node is special), and is uniquely responsible for managing the
local copy of the lockset.

AJFS uses a conceptually similar operation queuing mechanism to that of Coda
\cite{KS92}. In Coda, client reads are serviced by the local preferred server,
but client's writes are stamped with a local server ID, and then forwarded on to
other servers.

AJFS is a much simpler system than Coda is, and AJFS doesn't ever use timestamps
of any variety. Moreover, AJFS is a fully-distributed networked file system,
while Coda is a true network-based, distributed file system. Coda's able to
ensure full causal ordering, even amid nodes leaving and rejoining. AJFS uses
none of the complexity that Coda does.
