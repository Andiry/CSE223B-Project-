
In today's world, people use a variety of computers from a variety of locations.
They demand that their files remain synchronized and accessible from wherever
they are. In this space, the winners have been cloud storage services such as
Dropbox, Google Drive, Microsoft SkyDrive, Apple iCloud and others.

These services meet the need fairly effectively. As long as the user has
internet access, files remain available and largely synchronized. However, these
benefits are not without cost. A user must trust the cloud services to keep
their data secure and private, and synchronizing their files is subject to the
potentially relatively low bandwidth links between the user and the cloud
service.

We present AJFS: a fully-replicated file system that supports multiple users and
limited simultaneous editing across hosts. AJFS is robust against nodes failing
and joining the system, and is designed with the Dropbox use case in mind.

AJFS is not meant to replace Dropbox, but is instead meant as a first step
towards facilitating users to share files under similar usage patterns as
Dropbox without having to trust a third party service provided. On the other
hand, it presently has no support for file system changes while a host is
offline or unable to access all other nodes participating.

AJFS is fully functional and presently emphasizes data safety and correctness
over performance, but future work could dramatically decrease the performance
gap between local file systems and the AJFS system.

