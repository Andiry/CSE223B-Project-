RPC support is handled by a dedicated thread running an Apache Thrift Non-blocking
server. Thrift was chosen for convenience and ease of getting started (due to
our familiarity with it).

We use a new RPC for each \textit{modifying} file system call (i.e.
\texttt{write()}), as well as for opening/closing, locking/unlocking, pinging,
the join procedure. All procedures use Thrift's "oneway" designation, except
where synchronous behavior is required (such as in \texttt{lock()},
\texttt{fsync()} and join procedure RPCs).

Thrift's "oneway" modifier simply means that the client doesn't wait for the
server to finish the operation before returning (and thus the RPC must be
\texttt{void}). Normally, Thrift does not guarantee that multiple "oneway" RPCs
are executed serially or in the proper order. This is because in a
multi-threaded Thrift server, the server may schedule RPCs to different worker
threads.

In addition to the normal arguments, most operations also require a host
operation counter. This counter allows remote operations to be idempotent. When
the Thrift server receives an RPC from a sending host with a counter value less
than or equal to that of a previous call, the server drops the operation instead
of acting on the request.

Since we must provide a causal ordering for operations from the same host, we
use a single-threaded version of Thrift's Non-blocking server. Since it can only
service a single request at a time, and since it does not (usually) interrupt an
execution of an RPC to service a new one, this server maintains the ordering
requirements that we need.

We use Thrift's Non-blocking server instead of Thrift's "SimpleServer". This is
because "SimpleServer" is only able to service a single connection at a time,
and will not go onto a new server until that connection is closed. As an
optimization, we chose to keep connections between the servers open between RPC
calls. This avoids the added overhead of initializing and closing a new TCP
connection for every RPC, but necessitated a new server in order to support
more than two nodes.

