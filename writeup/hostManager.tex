The Host Manager is responsible for maintaining the connections to all other
known hosts. Every few seconds (currently configured to 2.5 seconds), the Host
Manager sends an asynchronous RPC 'ping' to each of the hosts believed to be
alive.

Even though these requests are "asynchronous", requests are queued on the
receiving server. As a result, it is determined immediately whether the
communication channel is still valid. If the request fails, that server is
marked as dead, and any locks that the remote host holds on the local machine
are forceably released and no further communication will be attempted with that
remote host unless it goes back through the join procedure.

By keeping track of what hosts succeed or fail, the Host Manager is also able to
verify that it still is able to talk with the majority of hosts. If it is not,
the Host Manager knows that either it has been partitioned into a minority
partition, or that more than half of the hosts in the group are down. In these
cases, the system exits, as it is not prepared to deal with requests in those
cases.

