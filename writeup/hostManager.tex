The Host Manager is a dedicated thread and is responsible for maintaining the
connections to all other known hosts. Every few seconds (currently configured to
2.5 seconds), the Host Manager sends an asynchronous RPC \texttt{ping()} to each
of the hosts believed to be alive. This \texttt{ping()} is designed to be
minimally intrusive, doing nothing more than returning immediately.

Even though these requests are "asynchronous" (Thrift's "oneway"), requests are
actually queued on the receiving server when they're initially sent. In this
way they're synchronously transmitted, but asynchronously executed. As a
result, we still must be able to successfully communicate with the remote
system immediately. If the request fails, that server is marked as dead, and
any file locks that the remote host holds on the local machine are forceably
released and no further communication will be attempted with that remote host
unless it goes back through the join procedure.

By keeping track of what hosts succeed or fail, the Host Manager also verifies 
that it is still in the majority of hosts. If it is not, the Host Manager knows
that either it has been partitioned into a minority partition, or that more
than half of the hosts in the group are down. In either of these cases, the
system exits.

The Host Manager is also responsible for timeouts on the "join lock". The
system maintains a configurable maximum allowed time period (currently 10
seconds) that a joining server can hold a "join lock" on the file system. This
corresponds to roughly the time between calls by a new server to
\texttt{requestJoinLock()} and \texttt{join()} on an already-joined server. If
the timeout expires before the "join lock" is released, we mark the new server
as dead (and refuse to respond to any subsequent RPC calls without
starting the join procedure over) and unlock the system.

Each server is responsible for resetting their own "join lock" (in case the
system that failed was the server facilitating the join). However, the
facilitating server has a "join lock" timeout slightly earlier than the others
to help prevent the possibility of a timeout occurring on one server, but having
\texttt{join()} start on the facilitator before the "join lock" times out there.
This case is safe, as \texttt{join()} will fail, but is undesirable as an
join procedure that would otherwise proceed must start again.

This period corresponds to the time given for the call to \texttt{rsync} to
complete. It is entirely possible that \texttt{rsync} may need more time to
sync the two systems, but we leave a better solution to future work.
It could be fixed by having a 'heartbeat' from the joining server sent
periodically to reset the timer. Our solution is still safe, and subsequently
likely to succeed, as eventually \texttt{rsync} will return. At this point, the
joining server will attempt to \texttt{join()} with the facilitating server.
This call will fail, but as \texttt{rsync} has just recently completed,
subsequent attempts to join will likely result in rsync completely more
quickly.


